\documentclass[11pt, oneside]{article}   	% use "amsart" instead of "article" for AMSLaTeX format
\usepackage{geometry}
\usepackage{enumitem}
             		% See geometry.pdf to learn the layout options. There are lots.
\geometry{letterpaper}                   		% ... or a4paper or a5paper or ...
%\geometry{landscape}                		% Activate for for rotated page geometry
%\usepackage[parfill]{parskip}    		% Activate to begin paragraphs with an empty line rather than an indent
\usepackage{graphicx}				% Use pdf, png, jpg, or eps§ with pdflatex; use eps in DVI mode
								% TeX will automatically convert eps --> pdf in pdflatex
\usepackage{amssymb}

\usepackage{stmaryrd}


\title{CS704: Assignment 2}
\author{}
\date{}							% Activate to display a given date or no date


\begin{document}
\maketitle
Questions 2 and 3 are adapted from TAPL.

%\begin{comment}
%\section{Galois connections}
%Consider these two definitions of Galois connections:

%\paragraph{Definition 1:}
%$(L,\alpha,\gamma,M)$ is a Galois connection
%between the complete lattices $(L,\sqsubseteq)$
%and $(M,\sqsubseteq)$ if and only if
%$\alpha:L \rightarrow M$ and $\gamma:M \rightarrow L$
%are monotone functions that satisfy the following conditions:
%$$\gamma \circ \alpha \sqsupseteq \lambda x. x$$
%$$\alpha \circ \gamma \sqsubseteq \lambda x. x$$
%where $\circ$ is function composition---that is,
%$\alpha \circ \gamma$ is the function that applies
%$\alpha$ \emph{to the result of} $\gamma$.

%\paragraph{Definition 2:}
%$(L,\alpha,\gamma,M)$ is a Galois connection
%between the complete lattices $(L,\sqsubseteq)$
%and $(M,\sqsubseteq)$ if and only if
%$\alpha:L \rightarrow M$ and $\gamma:M \rightarrow L$
%are total functions such that for all $l\in L, m \in M$,
%$$\alpha(l) \sqsubseteq m \Leftrightarrow l \sqsubseteq \gamma (m)$$

%~\\
%Prove that the two definitions are equivalent.
%\end{comment}

\section{Fixed-point combinators}

\begin{enumerate}[label=\bf\Alph*]
\item The most popular encoding of recursion in $\lambda$-calculus
    is using the Y combinator.
    Another approach is using the simple U combinator (which is
    not a fixed-point combinator):
    $$U =_\textrm{df} \lambda x.\ x\ x$$
    (We use $=_\textrm{df}$ to denote a definition, as opposed to
    semantic equivalence.)
    Using the U combinator, we can define the factorial
    function as follows:
    $$\emph{fact} =_\textrm{df} U(\lambda f.\ \lambda n.\ ~\emph{if}~(n=0)~\emph{then}~1~\emph{else}~n*((f f) (n-1)))$$
    (For clarity, we extend pure lambda calculus
    with conditionals and arithmetic.)\\
    Prove that $\emph{fact}$ satisfies
    the following $\lambda$-calculus equation:
    $$\emph{fact} = (\lambda n. ~\emph{if}~(n=0)~\emph{then}~1~\emph{else}~n*(\emph{fact} (n-1)))$$

\item
Consider the following theorem for characterizing
fixed-point combinators themselves as fixed points:
\begin{equation}
  \label{Eq:FPTheoremStmt}
  \textrm{Let $G = \lambda y . \lambda f . f(yf)$.
  Then $M$ is a fixed-point combinator if and only if $M = GM$.}
\end{equation}

\noindent
(Note: Recall that the following $\lambda$-calculus transformation
is called the $\eta$-reduction rule:
\[
  (\lambda x.Mx) \rightarrow_\eta M,
\]
where $x$ does not occur as one of the free variables of $M$.
You are allowed to use $\eta$-reduction in this question.)

\medskip

\noindent
{\bf subpart (i)}

\noindent
Use Theorem \ref{Eq:FPTheoremStmt} to show that
$Y =_\textrm{df} \lambda f . (( \lambda x.f(xx) )( \lambda x.f(xx) ))$
is a fixed-point combinator.

\medskip

\noindent
{\bf subpart(ii)}

\noindent
Use Theorem~\ref{Eq:FPTheoremStmt} to show
that the U combinator ($\lambda x.\ x\ x$)
is a not a fixed-point combinator.

\noindent
{\bf subpart (iii)}

\noindent
Prove Theorem \ref{Eq:FPTheoremStmt}.
(Note that the theorem involves an ``if and only if'';
consequently, your proof should have two parts.)

\noindent
{\bf subpart (iv)}

The Y combinator
allows us to find a $\lambda$-term $g$
that satisfies a single recursive equation over $\lambda$-terms of the
form $g = \ldots g \ldots g \ldots$

Suppose that we are presented with a collection of $k$ mutually recursive
equations:
\[
  \begin{array}{rcl}
    g_1 & = & \ldots g_1 \ldots g_k \ldots \\
        & \vdots & \\
    g_k & = & \ldots g_1 \ldots g_k \ldots
  \end{array}
\]
Explain how to solve for $g_1, \ldots, g_k$.


\end{enumerate}

\section{Church Encodings}
\begin{enumerate}[label=\bf\Alph*]
  \item In lecture, we defined Church numerals and Booleans, along with some  operations over them.
Define a lambda term that tests equality of two Church numerals,
returning $\lambda t. \lambda f. t$ when they are equal,
and $\lambda t. \lambda f. f$ when they are not equal.

\item Suppose you are given a combinator $\textsf{pred}$
that computes the predecessor of a Church numeral.
Use $\textsf{pred}$ to define subtraction in lambda calculus.
(Note that, given 0, $\textsf{pred}$ returns 0.)
\end{enumerate}

\section{Typed Lambda Calculus}
   Is there any context $\Gamma$ and type $\tau$
  such that $\Gamma \vdash x\ x : \tau$. If so,
  provide a $\Gamma$ and $\tau$ and show the typing derivation;
  if not, prove that there does not exist
  such a $\Gamma$ and $\tau$.
\end{document}
