\documentclass{amsart}

\usepackage{hyperref,url,amsmath,amssymb,proof,stmaryrd,tikz-cd,mathabx,prooftree}

\newtheorem{theorem}{Theorem}[section]
\newtheorem{lemma}[theorem]{Lemma}

\theoremstyle{definition}
\newtheorem{definition}[theorem]{Definition}
\newtheorem{example}[theorem]{Example}
\newtheorem{xca}[theorem]{Exercise}

\theoremstyle{remark}
\newtheorem{remark}[theorem]{Remark}

\numberwithin{equation}{section}

\newcommand{\ifst}{\mathsf{if}}
\newcommand{\thenst}{\mathsf{then}}
\newcommand{\elsest}{\mathsf{else}}

\newcommand{\opsem}[2]{\langle #1 \rangle \to #2}
\newcommand{\trans}{\emph{trans}}
\newcommand{\enc}{\textsf{enc}}
\newcommand{\fv}{\textrm{FV}}
\renewcommand{\paragraph}[1]{\vspace{.08in}\noindent\textbf{#1}~~}


\begin{document}


\title{Unbounded Verification with Horn Clauses}

\author{Aws Albarghouthi}
\address{University of Wisconsin--Madison}
\email{aws@cs.wisc.edu}

\maketitle

\begin{abstract}
These notes describe the process
of encoding program executions as a formula
over \emph{constrained Horn clauses},
a class of first-order logic formulas.
Horn clauses allow us to encode the search
for a Hoare-logic annotation of the program
that proves its correctness.
\end{abstract}

\section{Motivating Example}
We begin with an example illustrating constrained Horn
clauses and how they can encode Hoare-logic proofs.

\paragraph{A Hoare-style Proof}
Consider the following program and the associated
Hoare triple:
\begin{align*}
& \{x > 0\}\\
& y \gets x;\\
&z \gets x + y\\
& \{z > 0\}
\end{align*}

To prove that the above Hoare triple is valid,
we follow the composition rule of Hoare-logic:

\begin{figure}[h]
  \centering
  \prooftree
       \{\phi\} P_1 \{\psi\} \quad\quad
       \{\psi\} P_2 \{\chi\}
  \justifies
  \{\phi\} P_1; P_2 \{\chi\}
  \using \textsc{Composition}
  \endprooftree
\end{figure}

Reading the rule upwards,
to prove the above Hoare triple,
we need to construct two valid Hoare triples
of the following form:
$$\{x> 0\} ~ y \gets x ~ \{r(x,y,z)\}$$
$$\{r(x,y,z)\} ~ z \gets x + y ~ \{z > 0\}$$
where $r(x,y,z)$ is some formula over $x,y,z$
that we need to discover.
One solution for
$r(x,y,z)$ is $x > 0 \land y > 0$.
This gives us the following two valid Hoare triples:
$$\{x> 0\} ~ y \gets x ~ \{x > 0 \land y > 0 \}$$
$$\{x > 0 \land y > 0\} ~ z \gets x + y ~ \{z > 0\}$$
which following the composition rule,
imply that our original Hoare triple is valid.

\paragraph{Encoding}
Above, we showed how to pose the search for a Hoare-logic
annotation
as a search for a relation $r(x,y,z)$ over program variables.
We can view $r(x,y,z)$ as a relation in first-order
logic, and generate a set of constraints whose
models give solutions of $r$.

Consider the following formulas, $C_1$ and $C_2$,
which encode the two Hoare triples above:

\begin{align*}
C_1 \triangleq&
  \forall V,V' \ldotp (x > 0 \land \enc(y = x)) \Longrightarrow r(x',y',z')\\
%
C_2 \triangleq&
  \forall V,V' \ldotp   \left(r(x,y,z) \land
  \enc(z = x + y)\right) \Longrightarrow r(x',y',z')
\end{align*}
Both of these formulas
are over the theory of linear integer arithmetic (LIA),
where we also have a relation symbol $r$,
a ternary relation over integers, i.e., in $\mathbb{Z}^3$.
Since there are no free variables in $C_1,C_2$,
a model for $m \models C_1 \land C_2$
will only give an interpretation for $r(x,y,z)$.

The two formulas $C_1$ and $C_2$
look very much like our encodings for checking Hoare triples
from the previous lecture.
By construction, any model $m$ of $C_1$
must be such that for any initial state where $x > 0$,
if we execute $y = x$, we end in a state in $r(x',y',z')$;
note that we apply the relation $r$ to the final-state variables.
So any interpretation of $r$ must result in a valid Hoare triple
$\{x> 0\} ~ y \gets x ~ \{r(x,y,z)\}$.

Recall that a model $m$ maps $r$ to a susbet of $\mathbb{Z}^3$,
we will only consider subsets of $\mathbb{Z}^3$
that we can write as formulas in LIA;
therefore, we cannot have a model, for instance,
that contains only prime numbers, a relation that is not representable in our simple theory of LIA.

One possible $m$ is the one that sets $r(x,y,z)$
to the formula $x > 0 \land y > 0$.
If we plug in this formula for occurrences of
$r$ in $C_1, C_2$, we get the following formulas:

\begin{align*}
m(C_1) \triangleq&
  \forall V,V' \ldotp (x > 0 \land \enc(y = x)) \Longrightarrow x' > 0 \land y' > 0\\
%
m(C_2) \triangleq&
  \forall V,V' \ldotp   \left(x > 0 \land y > 0 \land
  \enc(z = x + y)\right) \Longrightarrow x' > 0 \land y' > 0
\end{align*}
Both formulas are valid.

\section{Constrained Horn Clauses}
A \emph{constrained Horn clause} $\cls$, or Horn clause for
short, is a first-order logic formula of the form
\[
\rel_1(\vec{v}_1) \land \rel_2(\vec{v}_2) \land \ldots \land \rel_{n-1}(\vec{v}_{n-1})
\land \varphi \Longrightarrow \head_\cls
\]
where:
\begin{itemize}
  \item each relation $\rel_i \in \rels$ is of arity  equal to the length of the
    vector of variables $\vec{v}_i$;
  \item $\varphi$ is a conjunction of atoms over the first-order theory, which contain non relation symbols outside those interpreted by the theory (e.g., $\phi$ could be $x > 0 \land y > 0$);
  \item the left-hand side of the implication ($\Longrightarrow$) is called the
    \emph{body} of $\cls$; and
  \item $H_\cls$, the \emph{head} of $\cls$, is either a relation application
    $\rel_n(\vec{v}_n)$ or an interpreted formula $\varphi'$.
  \item  All free variables
  are assumed to be universally quantified, e.g., $x+y > 0
  \Longrightarrow \rel(x,y)$ means $\forall x,y \ldotp x + y > 0
  \Longrightarrow \rel(x,y)$.
\end{itemize}

%
%\jh{In the interpretation, yes? So with $\Longrightarrow$?}
%


\paragraph{Semantics.}
We will write $\clss$ for a set of clauses $\{\cls_1,\ldots,\cls_n\}$.
Let $G$ be a graph over relation symbols such that
there is an edge $(r_1,r_2)$ iff $r_1$ appears in the body of
some clause $\cls_i$ and $r_2$ appears in its head.
We say that $\clss$ is \emph{recursive} iff $G$ has a cycle.
The set $\clss$ is \emph{satisfiable} if
there exists an interpretation $\interp$ of relation symbols $\rel_i$ such that every clause
$\cls \in \clss$ is valid.
We say that $\interp$ satisfies $\clss$ (denoted $\interp\models \bigwedge_{\cls_i \in \clss} \cls_i$)
iff for all
$\cls \in \clss$, $\interp \cls$ is valid (i.e., equivalent to $\emph{true}$),
where $\interp \cls$ is $\cls$ with every relation application $\rel(\vec{v})$
replaced by its interepretation in $m$, which we denote as $\interp(\rel(\vec{v}))$.


\section{Constructing Horn Clauses from Programs}
We now consider programs $P$
of the form:
$$P_\pre; \whilest~ b ~\dost~ P_\body$$
We would like to show that a Hoare triple
$\{\phi\} P \{\psi\}$ is valid.
To do so, we will generate a number of Horn clauses
whose solution is an inductive loop invariant:

\begin{align*}
  C_1 \triangleq&
  (\phi \land \enc(P_\pre)) \Longrightarrow \inv(V') & \text{initiation}\\
  %
  C_2 \triangleq&
  \left(\inv(V) \land b \land
    \enc(P_\body)\right) \Longrightarrow \inv(V') & \text{consecution}\\
%
  C_3 \triangleq&
    \inv(V) \land \neg b \Longrightarrow \psi & \text{safety}
\end{align*}

A model $m \models C_1\land C_2 \land C_3$
gives an interpretation of $\inv$ that is an inductive loop invariant.

The encoding for our program model is sound and complete.

\begin{theorem}[Soundness]
Let $m \models C_1 \land C_2 \land C_3$.
The predicate $m(\inv)$ is an inductive loop invariant.
\end{theorem}

\begin{theorem}[Completeness]
If $C_1 \land C_2 \land C_3$ is unsatisfiable,
then $\{\phi\} ~ P ~ \{\psi\}$ is not a valid Hoare triple.
\end{theorem}

While the above theorems assure that our encoding
is correct; in practice, checking satisfiability of
constrained Horn clauses in LIA is undecidable.

\begin{example}
Consider the following program from your assignment:
\begin{align*}
&\{x \geq 0 \land y > 0\}\\
& r \gets x;\\
& q \gets 0;\\
& \whilest \ r \geq y \ \dost\\
& \ \ \ \ r \gets r - y;\\
& \ \ \ \ q \gets q + 1;\\
&\{x = y * q + r \land 0 \leq r < y\}
\end{align*}

We show the encoding below:
\begin{align*}
  C_1 \triangleq&
  x \geq 0 \land y > 0 \land \enc(P_\pre) \Longrightarrow \inv(x',y',r',q') & \text{initiation}\\
  %
  C_2 \triangleq&
  \inv(x,y,r,q) \land r\geq y \land
    \enc(P_\body) \Longrightarrow \inv(x',y',r',q') & \text{consecution}\\
%
  C_3 \triangleq&
    \inv(x,y,r,q) \land  r < y \Longrightarrow x = y * q + r \land 0 \leq r < y
 & \text{safety}
\end{align*}
where $\enc(P_\pre)$ and $\enc(P_\body)$ are as described in the previous
lectures.
\end{example}

\section{Constructing Horn Clauses from Arbitrary Programs}
\paragraph{Loops}
In the previous section, we saw how to encode the verification
problem for programs with a single loop. We now consider programs
with an arbitrary number of loops.

To do so, we assume each statement $P$ in the program has a unique
line number, denoted by $\ell(P)$, and a child or two, $\ell_1(P)$
and $\ell_2(P)$, denoting the true and false branches of a while loop
or if statement.
The following definition of $\enchorn$ demonstrates
how to take a program statement and encode it as a set of Horn clauses.
Note that if and while statements require two Horn clauses,
one for each possible branch taken.
%
\begin{align*}
  \enchorn(x \gets a) \triangleq&  \inv_i(V) \land \enc(x \gets a) \Longrightarrow \inv_j(V')\\
  \enchorn(\ifst \ b \ \thenst \ P_1 \ \elsest \ P_2) \triangleq&
    \{\inv_i(V) \land b \Longrightarrow \inv_j(V), \\
    & \inv_i(V) \land \neg b \Longrightarrow \inv_k(V)\}\\
    \enchorn(\whilest \ b \ \dost \ P_1) \triangleq&
      \{\inv_i(V) \land b \Longrightarrow \inv_j(V), \\
      & \inv_i(V) \land \neg b \Longrightarrow \inv_k(V)\}
\end{align*}
where above $\ell(P) = i$, $\ell_1(P) = j$, and $\ell_2(P) = k$,
for every case of $P$ considered.

Now, given a program $P$, we apply $\enchorn$ to every statement
in $P$, i.e., to every while statement, if statement, and assignment
statement, and collect all Horn clauses in a set.
Let $\clss$ be the set of Horn clauses collected for $P$.
Assume that $P$'s first statement is labeled $\init$ (entry) and last statement
is labeled $\ret$ (exit).
Then, to prove a Hoare triple $\{\phi\} \ P  \ \{\psi\}$,
we solve the following Horn clauses:
%
$$ \clss \cup \{\phi \Longrightarrow \inv_\init(V), \  \inv_\ret(V) \Longrightarrow \psi\}$$
%
The two additional Horn clauses signify that the set of initial states
must be in the invariant at statement $\init$, and the
invariant at the exit statement must be in $\psi$.

\paragraph{Recursion} We now consider the problem of
encoding programs with procedures.
Given a procedure $f$ that takes one input and returns
one output, we encode the procedure as a transition
relation $f(x,y)$, where $x$ denotes the input to
the function and the $y$ the output. This transition relation
is called a \emph{function summary}, as it captures the effect
of the input--output behavior of a function while hiding the internal
computation.

For simplicity, we show how to perform the encoding for a single
recursive function, $f$, by redefining the encoding function
$\enc(f)$ as follows:

\begin{align*}
  \enc(x \gets a) \triangleq&~  x' = a \land \bigwedge_{y \neq x, y \in V} y' = y\\
  \enc(\ifst~ b ~ \thenst~ P_1 ~\elsest~ P_2) \triangleq&~
    (b \Longrightarrow \enc(P_1)) \land (\neg b \Longrightarrow \enc(P_2))\\
  \enc(P_1; P_2) \triangleq&~ \exists V'' \ldotp \trans_1(V,V'') \land \trans_2(V'',V')\\
  & \text{where } \trans_1(V,V') \equiv \enc(P_1)\\
        & \hspace{3em} \trans_2(V,V') \equiv \enc(P_2)\\
  \enc(y \gets f(x)) \triangleq&~ f(x,y') \land \bigwedge_{z \neq y, z \in V} z' = z\\
\end{align*}

The following Horn clause encodes the transition relation $f(x,y)$:

Now, suppose we want to show the that following
Hoare triple is valid:

$$ \{ \phi \} \ y \gets f(x) \ \{ \psi \}$$

We encode the following Horn clauses:
\begin{align*}
\enc(f) \Longrightarrow f(x,y)  \\
\phi \land f(x,y) \Longrightarrow \psi
\end{align*}
The first Horn clause encodes the relation $f(x,y)$;
the second encodes the Hoare triple, using the transition relation for
$f$.

\begin{example}
Consider the following popular recursive function, called McCarthy 91:
\begin{align*}
& mc(p):\\
& \ifst\ p > 100\\
& \ \ \ r \gets p - 10\\
& \elsest\\
& \ \ \ p1 \gets p + 11\\
& \ \ \ p2 \gets mc(p1)\\
& \ \ \ r \gets mc(p2)
\end{align*}
where $r$ is the return value.

We want to show the following Hoare triple:
$$\{\emph{true}\} \ r \gets mc(p) \ \{r \geq 91\}$$

Using the above encoding, we get:

\begin{align*}
p > 100 \land r = p - 10 \Longrightarrow& mc(p,r)\\
p \leq 100 \land p1 \gets p + 11 \land mc(p1,p2) \land mc(p2,r) \Longrightarrow& mc(p,r)\\
\emph{true} \land mc(p,r) \Longrightarrow& r \geq 91
\end{align*}

One solution for the above is the interpretation that
sets $mc(p,r)$ to the following relation:
$$ mc(p,r) \equiv r \geq 91$$
In other words, in order to prove that that return value of
$mc$ is always greater than or equal to 91, all we have to do
is to assume the summary that it always returns a value greater than
or equal to 91, regardless of the output.

\end{example}

\section{Solving Constrained Horn Clauses}
To be continued...

\end{document}
