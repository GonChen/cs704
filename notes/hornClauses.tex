\documentclass{amsart}

\usepackage{hyperref,url,amsmath,amssymb,proof,stmaryrd,tikz-cd,mathabx,prooftree}

\newtheorem{theorem}{Theorem}[section]
\newtheorem{lemma}[theorem]{Lemma}

\theoremstyle{definition}
\newtheorem{definition}[theorem]{Definition}
\newtheorem{example}[theorem]{Example}
\newtheorem{xca}[theorem]{Exercise}

\theoremstyle{remark}
\newtheorem{remark}[theorem]{Remark}

\numberwithin{equation}{section}

\newcommand{\ifst}{\mathsf{if}}
\newcommand{\thenst}{\mathsf{then}}
\newcommand{\elsest}{\mathsf{else}}

\newcommand{\opsem}[2]{\langle #1 \rangle \to #2}
\newcommand{\trans}{\emph{trans}}
\newcommand{\enc}{\textsf{enc}}
\newcommand{\fv}{\textrm{FV}}
\renewcommand{\paragraph}[1]{\vspace{.08in}\noindent\textbf{#1}~~}


\begin{document}


\title{Unbounded Verification with Horn Clauses}

\author{Aws Albarghouthi}
\address{University of Wisconsin--Madison}
\email{aws@cs.wisc.edu}

\maketitle

\begin{abstract}
These notes describe the process
of encoding program executions as a formula
over \emph{constrained Horn clauses},
a class of first-order logic formulas.
Horn clauses allow us to encode the search
for a Hoare-logic annotation of the program
that proves its correctness.
\end{abstract}

\section{Motivating Example}
We begin with an example illustrating constrained Horn
clauses and how they can encode Hoare-logic proofs.

\paragraph{A Hoare-style Proof}
Consider the following program and the associated
Hoare triple:
\begin{align*}
& \{x > 0\}\\
& y \gets x;\\
&z \gets x + y\\
& \{z > 0\}
\end{align*}

To prove that the above Hoare triple is valid,
we following the composition rule of Hoare-logic:

\begin{figure}[h]
  \centering
  \prooftree
       \{\phi\} P_1 \{\psi\} \quad\quad
       \{\psi\} P_2 \{\chi\}
  \justifies
  \{\phi\} P_1; P_2 \{\chi\}
  \using \textsc{Composition}
  \endprooftree
\end{figure}

Reading the rule upwards,
to prove the above Hoare triple,
we need to construct two valid Hoare triples
of the following form:
$$\{x> 0\} ~ y \gets x ~ \{r(x,y,z)\}$$
$$\{r(x,y,z)\} ~ z \gets x + y ~ \{z > 0\}$$
where $r(x,y,z)$ is some formula over $x,y$
that we need to discover.
One solution for
$r(x,y,z)$ is $x > 0 \land y > 0$.
This gives us the following two valid Hoare triples:
$$\{x> 0\} ~ y \gets x ~ \{x > 0 \land y > 0 \}$$
$$\{x > 0 \land y > 0\} ~ z \gets x + y ~ \{z > 0\}$$
which following the composition rule,
imply that our original Hoare triple is valid.

\paragraph{Encoding}
Above, we showed how to pose the search for a Hoare-logic
annotation
as a search for a relation $r(x,y,z)$ over program variables.
We can view $r(x,y,z)$ as a relation in first-order
logic, and generate a set of constraints whose
models give solutions of $r$.

Consider the following formulas, $C_1$ and $C_2$,
which encode the two Hoare triples above:

\begin{align*}
C_1 \triangleq&
  \forall V,V' \ldotp (x > 0 \land \enc(y = x)) \Rightarrow r(x',y',z')\\
%
C_2 \triangleq&
  \forall V,V' \ldotp   \left(r(x,y,z) \land
  \enc(z = x + y)\right) \Rightarrow r(x',y',z')
\end{align*}
Both of these formulas
are over the theory of linear integer arithmetic (LIA),
where we also have a relation symbol $r$,
a ternary relation over integers, i.e., in $\mathbb{Z}^3$.
Since there are no free variables in $C_1,C_2$,
a model for $m \models C_1 \land C_2$
will only give an interpretation for $r(x,y,z)$.

The two formulas $C_1$ and $C_2$
look very much like our encodings for checking Hoare triples
from the previous lecture.
By construction, any model $m$ of $C_1$
must be such that for any initial state where $x > 0$,
if we execute $y = x$, we end in a state in $r(x',y',z')$;
note that we apply the relation $r$ to the final-state variables.
So any interpretation of $r$ must result in a valid Hoare triple
$\{x> 0\} ~ y \gets x ~ \{r(x,y,z)\}$.

Recall that a model $m$ maps $r$ to a susbet of $\mathbb{Z}^3$,
we will only consider subsets of $\mathbb{Z}^3$
that we can write as formulas in LIA;
therefore, we cannot have a model, for instance,
that contains only prime numbers, a relation that is not representable in our simple theory of LIA.

One possible $m$ is the one that sets $r(x,y,z)$
to the formula $x > 0 \land y > 0$.
If we plug in this formula for occurrences of
$r$ in $C_1, C_2$, we get the following formulas:

\begin{align*}
m(C_1) \triangleq&
  \forall V,V' \ldotp (x > 0 \land \enc(y = x)) \Rightarrow x' > 0 \land y' > 0\\
%
m(C_2) \triangleq&
  \forall V,V' \ldotp   \left(x > 0 \land y > 0 \land
  \enc(z = x + y)\right) \Rightarrow x' > 0 \land y' > 0
\end{align*}
Both formulas are valid.

\section{Constrained Horn Clauses}

\section{Constructing Horn Clauses from Programs}
We now consider programs $P$
of the form:
$$P_\pre; \whilest~ b ~\dost~ P_\body$$
We would like to show that a Hoare triple
$\{\phi\} P \{\psi\}$ is valid.
To do so, we will generate a number of Horn clauses
whose solution is an inductive loop invariant:

\begin{align*}
  C_1 \triangleq&
  (\phi \land \enc(P_\pre)) \Rightarrow \inv(V') & \text{initiation}\\
  %
  C_2 \triangleq&
  \left(\inv(V) \land b \land
    \enc(P_\body)\right) \Rightarrow \inv(V') & \text{consecution}\\
%
  C_3 \triangleq&
    \inv(V) \land \neg b \Rightarrow \psi & \text{safety}
\end{align*}

A model $m \models C_1\land C_2 \land C_3$
gives an interpretation of $\inv$ that is an inductive loop invariant.

The encoding for our program model is sound and complete.

\begin{theorem}[Soundness]
Let $m \models C_1 \land C_2 \land C_3$.
The predicate $m(\inv)$ is an inductive loop invariant.
\end{theorem}

\begin{theorem}[Completeness]
If $C_1 \land C_2 \land C_3$ is unsatisfiable,
then $\{\phi\} ~ P ~ \{\psi\}$ is not a valid Hoare triple.
\end{theorem}

While the above theorems assure that our encoding
is correct; in practice, checking satisfiability of
constrained Horn clauses in LIA is undecidable.


\section{Solving Constrained Horn Clauses}

\end{document}
